\documentclass[11pt,a4paper]{article}

\usepackage{a4wide}
\usepackage{polski}

\usepackage[left=2cm,right=2cm,top=2cm,bottom=3cm]{geometry}

\usepackage[utf8]{inputenc} 

\usepackage{amsmath}
\usepackage{graphicx}

\usepackage{float}
\usepackage{relsize}
\usepackage{subcaption}
\usepackage{tabularx}

\usepackage[colorlinks=true, allcolors=blue]{hyperref}

\usepackage[skip=10pt plus1pt, indent=0pt]{parskip}
\usepackage{indentfirst}

\title{Bazy danych 2 \\
Dokumentacja projektu}
\author{Dominika Boguszewska \\
Piotr Lenczewski \\
Michał Machnikowski \\
Jakub Pęk \\
Tomasz Truszkowski}
\date{Semestr 24L}

\begin{document}

\maketitle

\section{Temat projektu}

Postanowiliśmy stworzyć aplikację służącą szukaniu osób w celu wspólnego uprawiania sportu. Użytkownicy mogą stworzyć wydarzenie, w którym deklarują miejsce, dzień oraz interesujący ich sport. Także mogą zapisać się na wcześniej utworzone wydarzenie po wyszukaniu takich, które ich interesują.

\section{Wymagania aplikacji}

\subsection{Wymagania funkcjonalne}

\begin{itemize}
    \item Użytkownik może stworzyć konto.
    \item Przy tworzeniu konta wymagane jest ustawienie silnego hasła.
    \item Po utworzeniu konta wysyłany jest email z potwierdzeniem do użytkownika.
    \item Użytkownik może zalogować się na swoje konto.
    \item Użytkownik może ustawić swoje zdjęcie profilowe.
    \item Użytkownik może ustawić opis swojego profilu.
    \item Użytkownik może dodać własne ogłoszenia z parametrami: rodzaj sportu, liczba poszukiwanych osób, miejsce, czas (jednorazowe czy cykliczne), dodatkowy opis.
    \item Użytkownik może wyszukiwać oferty według podanych kryteriów (np. rodzaj sportu, miejsce, czas).
\end{itemize}
 
\subsection{Wymagania niefunkcjonalne}

\begin{itemize}
    \item Aplikacja działa na systemie operacyjnym Linux.
    \item Aplikacja działa na systemie operacyjnym Windows.
    \item Każda strona aplikacji musi się załadować w przeciągu 2 sekund.
    \item Aplikacja powinna wytrzymać X użytkowników bez strat jakości.
    \item Email wysłany do użytkownika po utworzeniu konta musi do niego dojść w przeciągu 10 minut.
\end{itemize}

\section{Model pojęciowy (E-R)}

\section{Model relacyjny bazy danych}

\section{Zdenormalizowany model relacyjny bazy danych}

\section{Wykorzystane technologie}

\begin{itemize}
    \item MySQL
    \item Java
    \item Spring Boot
    \item Next.js
\end{itemize}

\section{Sposób uruchomienia aplikacji}

\end{document}
